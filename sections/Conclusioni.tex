\section*{Conclusioni}

Con il mio studio incentrato sul linfoma di non-Hodgkin, ho voluto portare alla luce le pratiche terapeutiche 
che vengono messe in atto per il trattamento della patologia e il nursing del paziente oncoematologico. 
Negli approcci terapeutici previsti per il paziente, che siano di terapia o di indagini diagnostiche, l'infermiere 
svolge un ruolo centrale.
Ogni approccio terapeutico che sia di radioterapia, chemioterapia, immunoterapia, trapianto o infusione delle CAR-T, 
ha delle complicanze associate e l'infermiere deve essere preparato a fronteggiarle. 
Bisogna sottolineare il ruolo centrale che gli infermieri hanno verso l’impiego delle CAR-T, nelle diverse fasi 
di trattamento: preparazione del paziente, infusione della terapia, post-infusione e follow-up, in particolare nella 
gestione delle complicanze.\\
Gli infermieri devono essere adeguatamente formati e aggiornati, bisogna fornire loro gli strumenti necessari per poter 
agire correttamente nella pratica clinica, poichè la tipologia di paziente e i trattamenti sono complessi. 
Il nursing inoltre, prevede di assicurarsi che il paziente, i familiari e la figura del caregiver, ricevano le 
giuste informazioni che riguardano test e procedure, nonchè istruzioni sulla gestione dell’accesso venoso centrale e le
potenziali complicanze che ne potrebbero derivare.
Agli infermieri, si consiglia di adottare nella pratica clinica, metodi di classificazione e valutazione oggettivi, 
come parte dell’agire quotidiano, soprattutto nei casi in cui si sospettano complicanze i
cui cambiamenti possono insorgere rapidamente nel tempo.\\
La gestione del dolore, è una delle dimensioni di cura più importanti, il ruolo degli infermieri è
fondamentale, dal punto di vista sia della valutazione, sia per quanto riguarda il trattamento farma-
cologico e non farmacologico. 
È essenziale la valutazione iniziale, ma è altrettanto indispensabile la rivalutazione
frequente del paziente, in quanto permette di determinare se gli interventi adottati sono stati
efficaci nel ridurre la sintomatologia.\\
L’infermiere è colui che stabilisce una relazione di aiuto costante e prolungata con il paziente 
oncologico e i suoi familiari, fornisce supporto psicologico ed emotivo, deve promuovere il ruolo del
caregiver e del nucleo familiare. Oltre a sapere e saper fare, deve saper essere, ovvero possedere quelle 
capacità comunicative e relazionali, insieme ad empatia e umanità, che sono proprie della sua figura;
però deve anche riuscire a sviluppare nel paziente una capacità di autocura e autogestione della malattia, 
che sia orientata verso l’autonomia.

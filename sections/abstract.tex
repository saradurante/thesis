\section*{Introduzione}
   
L’elaborato di tesi affronta il quadro patologico dei linfomi, con un focus sul linfoma non-Hodgkin e mette in risalto 
il ruolo fondamentale svolto dall’infermiere nei confronti del paziente oncoematologico.
La scelta verso l’argomento è stata dettata, dall’esperienza di tirocinio clinico svolto presso l’U.O. DH di Ematologia 
dell’Ospedale Le Scotte di Siena, in cui ho avuto l’opportunità di apprendere l’assistenza infermieristica al fianco del 
paziente oncoematologico, dalla sua presa in carico fino alla dimissione; ho arricchito le mie conoscenze riguardo la 
terapia farmacologica, gli schemi terapeutici e il particolare iter  terapeutico, che vengono svolti in questa Unità Operativa.\\
L’obiettivo dello studio è quello di analizzare la neoplasia a 360 gradi, ma vuole essere anche di aiuto per pazienti e 
caregiver riguardo la prevenzione delle complicanze e la loro gestione. Nonostante sia una patologia ampiamente conosciuta e 
discussa, il linfoma, non riceve forse l’attenzione che merita sotto il profilo dell’assistenza infermieristica e delle 
competenze che un infermiere deve possedere per approcciarsi a questo tipo di paziente.\\
I linfomi, sono tumori maligni, che originano dalla proliferazione incontrollata dei linfociti (B e T/NK) o delle cellule 
staminali linfoidi creando accumuli patologici principalmente all’interno dei linfonodi, ma possono interessare 
potenzialmente tutti gli altri organi. Il linfoma di non-Hodgkin, è una neoplasia comune in Italia, rappresenta la quinta 
forma di cancro negli uomini e la sesta nelle donne; nonostante l’età media di insorgenza sia intorno ai 60 anni, è una 
patologia che colpisce tutte le età. La causa che determina lo sviluppo del linfoma di non-Hodgkin è una mutazione del DNA del 
linfocita, ma ad oggi, non è ancora chiaro il motivo per cui tale mutazione avviene. Nonostante  ciò, i fattori di rischio 
sono molteplici: infezioni causate da alcuni virus e batteri, così come alcune malattie croniche, si possono ritenere 
responsabili dell’aumento del rischio dello sviluppo di alcuni sottotipi di linfoma.\\
Sono molteplici gli approcci terapeutici al linfoma, quale sia quello più adatto dipende da una serie variabile di fattori 
come ad esempio il sottotipo di linfoma, lo stadio della malattia, l’eventuale presenza di “sintomi B” e il 
possibile coinvolgimento extranodale. Gli approcci "tradizionali", prevedono l’utilizzo di radioterapia e chemioterapia, 
il trapianto autologo o allogenico di cellule staminali emopoietiche e l’immunoterapia; sono trattamenti in genere combinati 
tra di loro, come parte di uno schema terapeutico da seguire per un certo numero di cicli, della durata di diversi mesi.\\
La terapia cellulare con CAR-T (Chimeric Antigen Receptor) è una forma di terapia genica impiegata per le forme 
refrattarie e recidivanti, i cui risultati più importanti sono stati rilevati per la leucemia linfoblastica acuta e per il 
linfoma non-Hodgkin (LNH). Nonostante i risultati positivi dati dalla terapia delle CAR-T, che dimostrano il raggiungimento 
di risposte rapide e durature nel tempo, non mancano aspetti di tossicità acuta, che possono evolvere in situazioni di 
severità, se non anche di fatalità.
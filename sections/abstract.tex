\begin{center}
   INTRODUZIONE
   
Con questo elaborato di tesi, ho voluto affrontare il quadro patologico dei linfomi, con un focus sul linfoma non-Hodgkin, 
mettendo in risalto il ruolo fondamentale svolto dall'infermiere, nei confronti del paziente ematoncologico.
La mia scelta verso l'argomento è stata dettata, oltre che da esperienze personali, dall'esperienza di tirocinio clinico 
svolto presso l'U.O. DH di Ematologia dell'Ospedale Le Scotte di Siena, in cui ho avuto l'opportunità di apprendere 
l'assistenza infermieristica al fianco del paziente oncoematologico, dalla presa in carico, fino alla dimissione; 
trattandosi di un regime ambulatoriale, il paziente svolgeva tutto nella mattinata, nonostante la permanenza era 
solo di qualche ora, ho avuto modo di vedere l'ambiente familiare che il paziente aveva creato, con le proprie abitudini, 
i propri spazi, il rapporto con tutto il team di cura; ho visto tanta sofferenza, ma anche tanta forza.
Grazie all'esperienza di tirocinio, ho arricchito le mie conoscenze circa farmaci, 
schemi terapeutici e pratiche terapeutiche, che non vengono eseguite di routine in altre realtà di reparto. 
L'intento è quello di analizzare la neoplasia a 360 gradi, perchè nonostante sia una patologia ampiamente discussa,
non riceve forse l'attenzione che merita, in particolare la competenza infermieristica che ho avuto modo di osservare 
in reparto; ma vuole essere anche una guida per pazienti e caregiver che si trovano ad affrontare la patologia, riguardo 
la prevenzione delle complicanze e la loro gestione.
L'elaborato di tesi è articolato in cinque capitoli; nel primo capitolo, ho voluto dare le nozioni di anatomia e 
fisiologia del sistema linfatico, quali le componenti e i sistemi che ne regolano il funzionamento, con una breve 
introduzione ai linfomi. Nel secondo capitolo, ho iniziato ad affrontare la patologia del linfoma non-Hodgkin, circa le 
cause che determinano lo sviluppo di tale neoplasia, i dati epidemiologici (incidenza della malattia, mortalità, 
sopravvivenza), i sintomi, le modalità di classificazione e stadiazione, gli indici prognostici e gli esami diagnostici. 
Nel terzo capitolo, ho intrapreso l'approfondimento dei diversi approcci terapeutici al linfoma di non Hodgkin, 
ho iniziato con le terapie tradizionali, come la radioterapia, l'immunoterapia, il trapianto di cellule staminali 
emopoietiche, per poi affrontare la chemioterapia nel quarto capitolo, in cui ho voluto descrivere i rischi cui è 
esposto il personale sanitario, che manipola i farmaci chemioterapici, gli accessi venosi utilizzati per il paziente 
oncologico e la gestione infermieristica degli stessi, nonchè le complicanze che possono verificarsi. 
L'innovativa terapia delle CAR-T è affrontata nel quinto e ultimo capitolo, con le  diverse 
fasi che la caratterizzano, le relative complicanze, il nursing e le CAR-T. In quest'ultima parte ho inoltre 
voluto affrontare la tematica del dolore nel paziente oncologico e la sua gestione, per finire con l'aspetto 
comunicativo tra infermiere e paziente. 

\end{center}
\chapter{Linfomi}

\section{Introduzione ai linfomi}
I linfomi sono tumori causati dalla proliferazione incontrollata dei linfociti.
All’interno del linfocita si ha la comparsa di mutazioni che coinvolgono un numero elevato di geni, 
implicati nella proliferazione, crescita e morte dei linfociti stessi.
I linfociti mutati vanno a invadere principalmente i linfonodi, creando degli accumuli patologici al loro interno, 
ma tale patologia può coinvolgere tutti gli organi.  
I linfomi si presentano con caratteristiche estremamente variabili, in virtù del fatto che le mutazioni 
possono insorgere in diverse fasi dello sviluppo del linfocita\cite{LINFOMIAIL}.\\

In base all’origine cellulare, si distinguono i linfomi che originano dalla linea T, 
più rari nelle popolazioni occidentali, e i linfomi che originano dalla linea B che sono i più frequenti. 
Distinguiamo i linfomi di Hodgkin, che originano dalla trasformazione dei linfociti B; 
i linfomi non-Hodgkin in cui sono coinvolte entrambe le tipologie di linfociti, sia B che T.\\

Le cause e i fattori di rischio che possono dare origine a tale malattia, sono in gran parte sconosciuti, 
è noto che infezioni causate da alcuni virus e batteri, così come alcune malattie croniche, 
si possono ritenere responsabili dell’aumento del rischio dello sviluppo di alcuni sottotipi di linfoma\cite{LINFOMIAIL}.\\

\section{Linfoma non-Hodgkin}
I linfomi non-Hodgkin (LNH) sono neoplasie che originano dalla proliferazione incontrollata dei linfociti B e T; 
tali cellule fanno parte del sistema immunitario e si trovano nel sangue, nel tessuto linfatico di linfonodi, milza, 
timo e midollo osseo\cite{LNHAIL}.\\ 
Nel LNH i linfociti creano accumuli patologici all’interno dei linfonodi. La differenza tra i LNH e il 
linfoma di Hodgkin è l’assenza della cellula di Reed-Sternberg, un particolare tipo di linfocita B caratterizzato 
dalla presenza di due nuclei, che permette di distinguere una cellula sana da una malata\cite{LNHAIMAC}.\\

\section{Cause}
Ad oggi non sono chiare le cause che portano alla comparsa di tale patologia, sappiamo che è la mutazione del 
DNA del linfocita che porta allo sviluppo di tale neoplasia, ma non sono ancora note le cause per cui tale 
mutazione si verifica.\\ 
Tuttavia, alcuni fattori aumentano il rischio di sviluppo di determinati sottotipi di linfoma come ad esempio 
l’immunodepressione, quindi l’ indebolimento del sistema immunitario, causato ad esempio da terapie con farmaci 
immunosoppressori usati nel post-trapianto o in caso di infezione da HIV, che indebolisce anch’esso 
il sistema immunitario.\\ 

Ma anche malattie autoimmuni, come il Lupus eritematoso sistemico (LES), artrite reumatoide, 
la malattia di Sjogren (Sjögren), la celiachia sono correlate ad un aumento del rischio di sviluppo del NHL, 
in quanto l’iperattività del sistema immunitario in questo tipo di malattie può provocare la crescita e la 
divisione dei linfociti in modo incontrollato, 
e ciò può comportare lo sviluppo di cellule anomale\cite{AMERICANCS}.\\

Sono fonte di rischio anche le infezioni virali croniche, in quanto il sistema immunitario, 
per combattere l’infezione, stimola una maggiore produzione di linfociti, che può aumentare il rischio che si 
verifichino mutazioni nel DNA del linfocita stesso. Esempi di infezioni virali croniche sono l’epatite C, 
la precedente esposizione al virus di Epstein-Barr (responsabile della mononucleosi infettiva), 
l'infezione da virus della leucemia umana a cellule T (HTLV-1), maggiormente diffuso in alcune parti del 
Giappone e nei Caraibi, ma si può trovare in tutto il mondo (la trasmissione è sessuale, mediante sangue infetto 
oppure tramite il latte materno da una madre infetta); l’infezione da virus dell'herpes umano 8 (HHV-8), 
può causare un linfoma molto raro, noto come linfoma a versamento primario, che si presenta in soggetti infetti da HIV.\\

Le infezioni batteriche, rappresentano anch’esse un fattore di rischio nello sviluppo del LNH. 
L’infezione causata da Helicobacter pylori, rappresenta la prima causa di linfoma primitivo dello stomaco; 
la Chlamydophila psittaci, è un batterio collegato allo sviluppo del linfoma nei tessuti intorno all’occhio, 
chiamato linfoma della zona marginale annessiale oculare; il batterio Campylobacter jejuni è correlato a un tipo 
di linfoma MALT, chiamato malattia immunoproliferativa dell'intestino tenue\cite{AMERICANCS}.\\
Inoltre, il trattamento con chemioterapia o radioterapia, a cui è sottoposta la persona a causa di altri tipi di tumore, 
può aumentare negli anni seguenti  il rischio di sviluppare il LNH\cite{AMERICANCS}.\\

L’esposizione a sostanze chimiche tossiche, come pesticidi e derivati del benzene, è un fattore di rischio. 
Il linfoma non Hodgkin non sembra essere presente in più membri di una stessa famiglia, tuttavia, 
il rischio può essere leggermente maggiore se un parente di primo grado (come un genitore o un fratello/sorella) 
ha avuto il linfoma\cite{AMERICANCS}.\\
Il linfoma può colpire a qualunque età, però c’è un maggiore rischio nella sesta decade di vita, 
maggiormente negli uomini piuttosto che nelle donne, ma alcuni sottotipi di LNH si manifestano 
maggiormente nelle donne, per cause non ancora note. 
Un maggiore rischio si presenta inoltre nei paesi industrializzati e nella razza bianca\cite{AMERICANCS}.\\ 

\section{Epidemiologia}
I linfomi non Hodgkin rappresentano globalmente il 4-5\% delle nuove diagnosi di neoplasia nella popolazione 
occidentale e in Italia 
sono la quinta forma di cancro più comune negli uomini e la sesta nelle donne\cite{AIOM}. 
L’età media di insorgenza è compresa tra i 50 e 60 anni e l’incidenza tende a incrementare con l’aumentare dell’età. 
Il LNH può tuttavia presentarsi ad ogni età.\\ 
In Italia si calcolano circa 19-20 nuovi casi per 100.000 abitanti ogni anno, pertanto 16.000 nuovi casi di linfoma, 
con un incremento annuo pari all’1,3\%\cite{AIOM}.\\
L’incidenza dei linfomi risulta essere influenzata da fattori geografici, razziali e dall'età, 
pertanto avremo una maggiore incidenza nei paesi industrializzati, negli uomini e nella razza bianca\cite{AIOM}.\\
L’incidenza è in aumento, secondo le stime dei Registri Tumori AIRTUM. “L’Associazione Italiana Registri Tumori 
(AIRTUM) nasce a Firenze nel 1996 con lo scopo di promuovere, coordinare e sostenere l’attività di registrazione 
dei tumori in Italia\cite{AIRTUM}.”\\
Per l’anno 2020 le stime parlano di 7000 nuovi casi tra gli uomini e 6100 tra le donne. 
Nonostante ciò, la mortalità resta stabile negli anni\cite{AIRC}.\\
In America, il LNH è uno dei tumori più diffusi, rappresenta circa il 4\% delle diagnosi di tumore. 
Secondo le stime dell’American Cancer Society, negli Stati Uniti nel 2022 sono previste circa 80.000 nuove 
diagnosi di LNH e circa 20.000 decessi\cite{Americanstatistic}.\\

\section{Classificazione}
In base alla modalità di crescita e progressione della malattia, i linfomi non-Hodgkin vengono suddivisi in 
linfomi indolenti o a basso grado di malignità, in cui c’è infatti una lenta progressione della malattia, 
non sono presenti sintomi sistemici e le dimensione dei linfonodi affetti aumentano lentamente e tendono a recidivare. 
I linfomi aggressivi o ad alto grado di malignità, sono invece caratterizzati da una rapida progressione della malattia, 
sono associati a sintomi sistemici e necessitano di trattamento tempestivo\cite{reteveneta}.\\

Sono diversi i sottotipi di LNH, attualmente la classificazione segue i criteri proposti dalla WHO 
(World Health Organisation), basati sulla precedente classificazione REAL (Revised European American Lymphoma) 
del 1994 che ha identificato più di 60 tipi di NHL, descritti come unità a sé stanti\cite{AIOM}.\\
La classificazione, identifica i LNH in base alla cellula di origine (linfocita B, T o NK), e quindi si basa su 
criteri morfologici, immunofenotipici, genetici e molecolari, integrati con le caratteristiche di presentazione clinica. 
La classificazione WHO riconosce pertanto tre principali categorie: linfomi a cellule B mature, linfomi a cellule T/NK
mature, linfomi di Hodgkin\cite{AIOM}.\\
Tra i linfomi a cellule B mature, i linfomi diffusi a grandi cellule B (indicati con l’acronimo DLBCL, 
diffuse large B cell lymphoma) sono linfomi aggressivi e da soli rappresentano il sottotipo più frequente, 
sono circa il 25-35\% dei LNH\cite{AIOM}.\\
Il linfoma follicolare (FL) è un linfoma a cellule B e rappresenta circa il 20\% dei NHL.\\

\section{Sintomi}
I linfomi, spesso, possono risultare asintomatici e alla diagnosi la malattia si presenta ad uno stadio già avanzato.
Generalmente il sintomo più frequente è l’ingrossamento persistente e non dolente dei linfonodi di ascelle, inguine e 
collo.\\ 
Sintomi sistemici invece possono essere febbre, perdita di peso, sudorazione notturna, prurito persistente; 
questi sintomi si manifestano quando sono coinvolti linfonodi profondi, che non possono quindi essere né osservati 
ne palpati e sono più comuni nel caso di linfomi aggressivi.\\ 
I linfomi possono inoltre causare epatomegalia e splenomegalia. 
Linfomi più aggressivi, possono colpire organi che non appartengono al sistema linfatico, come stomaco, intestino, 
cute, sistema nervoso, mammella, testicolo, e in tal caso i sintomi sono specifici, cioè dipendono 
dall’organo interessato.\\ 
Le cellule tumorali possono anche invadere il sangue, in questo caso si parla di leucemizzazione periferica.

\section{Diagnosi}
alibaba


\chapter{Sistema Linfatico}

\section{Cenni di Anatomia e Fisiologia}
Il sistema linfatico \`e una componente essenziale del sistema immunitario, 
insieme a milza, midollo osseo, timo, tonsille, appendice ciecale e placche di Peyer. 
Esso si compone di organi linfatici e vasi linfatici e rappresenta 
un sistema di drenaggio parallelo a quello venoso.\\
Gli organi linfatici sono suddivisi in primari o centrali 
e secondari o periferici; gli organi linfatici primari sono midollo osseo 
e timo, al loro interno i linfociti maturano per diventare linfociti T (nel timo) 
o linfociti B (nel midollo osseo). 
Gli organi linfatici secondari sono linfonodi, milza, e MALT 
(tessuto linfoide associato alle mucose) costituito da tonsille, 
placche di Peyer, appendice ciecale e altri raggruppamenti linfocitari 
sparsi nelle mucose. Gli organi linfoidi secondari, hanno un ruolo fondamentale 
nella risposta immunitaria, in quanto, i linfociti attivati, svolgono le loro funzioni 
in seguito al contatto con l'antigene.

I vasi linfatici possiedono strutture valvolari che garantiscono un flusso monodirezionale; 
il sistema linfatico infatti non possiede una pompa centrale, 
come avviene per il sistema circolatorio, ma la linfa scorre grazie ad una motricit\'a spontanea delle pareti.

Il sistema linfatico presenta delle interruzioni lungo il suo decorso: i linfonodi. 
I linfonodi filtrano e depurano la linfa dai microrganismi, dalle cellule neoplastiche e 
da particelle estranee. Essi si raggruppano nelle aree da cui si diramano i vasi linfatici, 
come collo, ascelle e inguine.

La linfa \'e il tessuto connettivo fluido trasportato e regolato da tale sistema, \'e costituita 
da liquidi che trasudano nei tessuti dell'organismo, tramite le pareti dei capillari; essa \'e costituita 
da ossigeno, proteine, globuli bianchi e altre sostanze nutrienti. 
La linfa trasporta sostanze estranee (batteri), cellule tumorali, cellule morte o danneggiate. 
I vasi linfatici pi\'u piccoli si collegano a quelli pi\'u grandi e formano il dotto linfatico e il dotto 
toracico, quest'ultimo \'e il vaso linfatico pi\'u grande, che si collega alla vena succlavia per poi restituire 
la linfa alla circolazione sanguigna. 

\section{Funzioni del Sistema Linfatico}
Le funzioni del sistema linfatico sono: produzione, mantenimento e distribuzione dei linfociti, 
mantenimento della volemia, via alternativa per il trasporto di ormoni, sostanze nutritizie e di scarto.


I linfociti sono indispensabili per i normali meccanismi di difesa, vengono prodotti e accumulati a livello degli organi linfoidi. 
A livello degli organi linfoidi primari, midollo osseo e timo, si ha lo sviluppo e la maturazione dei linfociti i quali successivamente
si differenzieranno in cellule B, T o NK (natural killer). 
La risposta immunitaria inizia a livello degli organi linfoidi secondari, che sono la polpa bianca della milza,
le tonsille, l'appendice, le placche di peyer e i linfonodi, dove in caso di infezione, ad esempio, i linfociti B
si dividono per produrre ulteriori linfociti B, necessari a debellare l'infezione.

Il sistema linfatico, ha un ruolo importante nel mantenimento della volemia (volume sanguigno) in quanto 
quotidianamente si ha un movimento di un notevole quantitativo di liquidi attraverso il sistema linfatico e la rottura 
di uno dei vasi linfatici può portare ad una rapida, nonché fatale, diminuzione della volemia.

Il sistema linfatico rappresenta inoltre una via alternativa per il trasporto di ormoni, 
sostanze nutritizie e di scarto in quanto alcune sostanze, come anche alcuni lipidi, vengono assorbiti
nel circolo sanguigno tramite i vasi linfatici, piuttosto che per assorbimento da parte delle pareti dei capillari.

\section{Struttura dei Vasi Linfatici}
\section{Linfociti}
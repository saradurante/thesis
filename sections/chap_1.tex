\chapter{Sistema Linfatico}

\section{Cenni di Anatomia e Fisiologia}
Il sistema linfatico \`e una componente essenziale del sistema immunitario, 
insieme a milza, midollo osseo, timo, tonsille, appendice ciecale e placche di Peyer. 
Esso si compone di organi linfatici e vasi linfatici e rappresenta 
un sistema di drenaggio parallelo a quello venoso.\\

Gli organi linfatici sono suddivisi in primari o centrali 
e secondari o periferici; gli organi linfatici primari sono midollo osseo 
e timo, al loro interno i linfociti maturano per diventare linfociti T (nel timo) 
o linfociti B (nel midollo osseo). 
Gli organi linfatici secondari sono linfonodi, milza, e MALT 
(tessuto linfoide associato alle mucose) costituito da tonsille, 
placche di Peyer, appendice ciecale e altri raggruppamenti linfocitari 
sparsi nelle mucose. Gli organi linfoidi secondari, hanno un ruolo fondamentale 
nella risposta immunitaria, in quanto, i linfociti attivati, svolgono le loro funzioni 
in seguito al contatto con l'antigene.\\

I vasi linfatici possiedono strutture valvolari che garantiscono un flusso monodirezionale; 
il sistema linfatico infatti non possiede una pompa centrale, 
come avviene per il sistema circolatorio, ma la linfa scorre grazie ad una motricit\'a spontanea delle pareti.\\

Il sistema linfatico presenta delle interruzioni lungo il suo decorso: i linfonodi. 
I linfonodi filtrano e depurano la linfa dai microrganismi, dalle cellule neoplastiche e 
da particelle estranee. Essi si raggruppano nelle aree da cui si diramano i vasi linfatici, 
come collo, ascelle e inguine.\\
La linfa \'e il tessuto connettivo fluido trasportato e regolato da tale sistema, \'e costituita 
da liquidi che trasudano nei tessuti dell'organismo, tramite le pareti dei capillari; essa \'e costituita 
da ossigeno, proteine, globuli bianchi e altre sostanze nutrienti. 
La linfa trasporta sostanze estranee (batteri), cellule tumorali, cellule morte o danneggiate. 
I vasi linfatici pi\'u piccoli si collegano a quelli pi\'u grandi e formano il dotto linfatico e il dotto 
toracico, quest'ultimo \'e il vaso linfatico pi\'u grande, che si collega alla vena succlavia per poi restituire 
cd zila linfa alla circolazione sanguigna. \\

\section{Funzioni del Sistema Linfatico}
Le funzioni del sistema linfatico sono: produzione, mantenimento e distribuzione dei linfociti, 
mantenimento della volemia, via alternativa per il trasporto di ormoni, sostanze nutritizie e di scarto.


I linfociti sono indispensabili per i normali meccanismi di difesa, vengono prodotti e accumulati a livello degli organi linfoidi. 
A livello degli organi linfoidi primari, midollo osseo e timo, si ha lo sviluppo e la maturazione dei linfociti i quali successivamente
si differenzieranno in cellule B, T o NK (natural killer). 
La risposta immunitaria inizia a livello degli organi linfoidi secondari, che sono la polpa bianca della milza,
le tonsille, l'appendice, le placche di peyer e i linfonodi, dove in caso di infezione, ad esempio, i linfociti B
si dividono per produrre ulteriori linfociti B, necessari a debellare l'infezione.

Il sistema linfatico, ha un ruolo importante nel mantenimento della volemia (volume sanguigno) in quanto 
quotidianamente si ha un movimento di un notevole quantitativo di liquidi attraverso il sistema linfatico e la rottura 
di uno dei vasi linfatici può portare ad una rapida, nonché fatale, diminuzione della volemia.

Il sistema linfatico rappresenta inoltre una via alternativa per il trasporto di ormoni, 
sostanze nutritizie e di scarto in quanto alcune sostanze, come anche alcuni lipidi, vengono assorbiti
nel circolo sanguigno tramite i vasi linfatici, piuttosto che per assorbimento da parte delle pareti dei capillari.



\section{I vasi linfatici}
La linfa che scorre nei capillari linfatici, viene raccolta da due gruppi di vasi linfatici:
i vasi linfatici superficiali e i vasi linfatici profondi. 
I vasi linfatici superficiali si trovano nel tessuto sottocutaneo, nei tessuti connettivi lassi
delle membrane sierose (cavità pleurica, pericardica e peritoneale) e mucose (apparati digerente, respiratorio,
urinario e riproduttivo). I vasi linfatici profondi raccolgono la linfa proveniente dalla muscolatura scheletrica,
da altri organi di collo, arti, tronco, dai visceri delle cavità toracica e addominopelvica.\\ 
Nel tronco, i vasi linfatici superficiali e profondi convergono a formare vasi di calibro maggiore,
i tronchi linfatici (includono i tronchi lombari, intestinali, broncomediastinici, succlavii, giugulari),
che a loro volta si svuotano in due grossi vasi collettori, i dotti linfatici, che convogliano
la linfa nella circolazione venosa.\\
Il dotto toracico, che è il vaso linfatico più grande, raccoglie la linfa da tutta
la porzione sottodiaframmatica del corpo e dalla metà sinistra sopradiaframmatica; 
si svuota nel punto di confluenza della vena succlavia sinistra e in tal modo, nel sistema venoso,
rientra la linfa proveniente dal lato sinistro di testa, collo, torace e dall'intera porzione sottodiaframmatica del corpo.\\
Il dotto linfatico destro ha un diametro relativamente ridotto e raccoglie la linfa dalla metà destra
del corpo superiore al diaframma. Essa si svuota nel punto in cui confluiscono la vena giugulare interna destra e la vena succlavia destra. 

\section{Linfociti}
I linfociti rappresentano il tipo cellulare principale del sistema linfatico
e sono deputati all'immunità specifica. Essi intervengono per eliminare agenti estranei
(virus, batteri, cellule neoplastiche, tossine batteriche) o per renderli inoffensivi. 
I principali tipi di linfociti sono le cellule T (timo dipendenti) e le cellule B (bone marrow-derived).\\
I linfociti T rappresentano circa l'80\% dei linfociti circolanti, essi originano dal midollo osseo,
poi migrano nel timo dove si differenziano e diventano immunocompetenti. Vi sono due sottoclassi
di linfociti T: i linfociti T-helper e i linfociti T citotossici o NK (natural killer).
I linfociti NK rappresentano il 5-10\% dei linfociti circolanti e
insieme ai macrofagi attivati, sono responsabili della sorveglianza immunitaria
grazie al controllo costante dei tessuti periferici.
I linfociti B rappresentano il 10-15\% dei linfociti circolanti, originano e diventano immunocompetenti
nel midollo osseo e la loro funzione principale è quella di trasformarsi in plasmacellule, per produrre e secernere anticorpi.

\section{Patologie del sistema linfatico: i linfomi}
I linfomi, sono tumori maligni, che originano dalla proliferazione incontrollata
dei linfociti o di cellule staminali linfoidi. Tale proliferazione incontrollata,
è causata dalla comparsa di mutazioni dei geni coinvolti nella maturazione,
nella crescita e nella morte dei linfociti, che possono infatti acquisire la capacità
di replicarsi in modo incontrollato. 
I linfomi presentano caratteristiche estremamente variabili, proprio in virtù delle molteplici
mutazioni che possono insorgere nelle diverse fasi dello sviluppo del linfocita. 
I linfociti si accumulano e invadono più frequentemente i linfonodi, ma possono localizzarsi anche in altri organi.\\

In base all'origine cellulare, si parla di due sottotipi di linfoma:
quelli che originano dalla linea B, i più frequenti, e quelli più rari per i paesi occidentali,
che sono quelli della linea T. 
I fattori scatenanti tale malattia sono in gran parte sconosciuti, tuttavia alcune infezioni
da parte di virus e batteri, nonché alcune malattie croniche, aumentano il rischio di sviluppare alcuni sottotipi di linfoma.
I linfomi vengono suddivisi in Linfomi di Hodgkin e Linfomi non-Hodgkin; i primi sono dovuti alla trasformazione
dei linfociti B, i secondi possono coinvolgere sia linfociti B che T.\\

Circa 4 persone ogni 100.000 abitanti contraggono il linfoma di Hodgkin
che rappresenta circa lo 0,5\% di tutti i casi di tumore diagnosticati.
Tuttavia è uno dei tumori più frequenti nella fascia di età tra i 15 e i 35 anni. 
La sua incidenza è in aumento, in Europa si è rilevato un incremento del 22\%
nel decennio dal 2003 al 2014 e sono stati diagnosticati circa 17.600 nuovi casi
nel 2012, e circa 66.000 nuovi casi nel mondo nello stesso anno, ma la mortalità è in diminuzione.

I linfomi non-Hodgkin (LNH) sono un gruppo eterogeneo di tumori che colpiscono
in genere la popolazione adulta e anziana e in Italia rappresentano circa il 3\% di tutte le neoplasie. 
L'incidenza è in aumento e le stime dei Registri Tumori AIRTUM per il 2020 parlano
di 7.000 nuovi casi tra gli uomini e di 6.100 tra le donne. Nonostante ciò,
la mortalità resta stabile negli anni, anche grazie ai progressi nelle terapie. 
Il LNH può colpire in linea teorica tutte le età, ma oltre la metà dei LNH riguarda persone con più di 65 anni.\\

Il primo sintomo solitamente associato a qualunque tipo di linfoma è un ingrossamento non dolente dei linfonodi,
che assumono una consistenza compatta e gommosa, e tale condizione viene solitamente ignorata
fino alla comparsa di sintomi secondari, come febbre ricorrente, perdita di peso,
sudorazione notturna, problemi gastrointestinali e respiratori; negli stati avanzati della malattia invece,
si può riscontrare epatomegalia, splenomegalia, polmonite, anemia, problemi a carico del SNC, alterazioni cutanee.\\

Le modalità di crescita e progressione della malattia sono un aspetto importante
per la classificazione dei linfomi, che possono essere linfomi indolenti se
a crescita lenta, linfomi aggressivi in caso di crescita rapida. 
La strategia terapeutica da attuare e la prognosi dipendono dalla tipologia di
malattia e dalla sua velocità di progressione.






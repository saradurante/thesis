\chapter{Chemioterapia}

\section{La chemioterapia nel linfoma non-Hodgkin}

Uno dei principali approcci terapeutici nel linfoma non-hodgkin è la chemioterapia, un trattamento che prevede 
l’utilizzo di farmaci, somministrati attraverso diverse vie, che, tramite il circolo sanguigno, raggiungono 
le cellule cancerose per distruggerle\cite{CHEMOUK}.\\
Proprio perché il farmaco chemioterapico circola nel torrente ematico, questo tipo di trattamento è detto sistemico. 
La chemioterapia, è impiegata con un intento curativo, cioè va a distruggere le cellule cancerose per ambire alla 
guarigione dalla malattia; può essere un’opzione di trattamento per controllare la patologia, cioè il cancro non 
viene eliminato, ma si tenta di impedire la sua ulteriore crescita e diffusione. Infine, può essere impiegata a scopo 
palliativo (quando il cancro è diffuso e non può essere controllato), come opzione per alleviare 
i sintomi quali dolore e nausea, nonché per migliorare la qualità di vita del paziente\cite{CHEMOAMERICAN}.\\
Il trattamento chemioterapico, in genere, prevede una serie di cicli di terapia in cui vengono combinati diversi 
farmaci, con lo scopo di andare ad attaccare e distruggere le cellule in diverse fasi del ciclo cellulare\cite{LYMPHACTION}.\\
La terapia complessivamente ha la durata di circa sei mesi, in cui si alternano giorni consecutivi di terapia, 
a periodi di riposo della durata di circa due settimane, per permettere al corpo di riprendersi dagli effetti 
collaterali che la terapia stessa comporta. Il numero di cicli di terapia dipende da diversi fattori: dal tipo di 
linfoma, dai farmaci che si stanno somministrando, da come la malattia sta rispondendo alla terapia, 
da come il corpo sta rispondendo agli effetti collaterali\cite{CHEMOUK}.\\

\subsection{Vie di somministrazione dei farmaci chemioterapici: orale, endovenosa, intratecale}

La chemioterapia può essere somministrata per via orale, endovenosa o intratecale.
La chemioterapia orale viene assunta mediante forme farmaceutiche come compresse o capsule, bisogna seguire le 
corrette indicazioni e prestare attenzione ad assumere il corretto dosaggio, rispettare il corretto orario di 
assunzione, nonché l’indicazione di assumerle prima o dopo i pasti\cite{LYMPHACTION},\cite{CHEMOUK}.\\
La chemioterapia endovenosa viene somministrata tramite catetere venoso periferico o mediante catetere venoso centrale. 
Il farmaco può essere somministrato attraverso dosi bolo, oppure tramite infusione con pompe infusionali, in modo da 
avere un maggior controllo della dose somministrata, con una durata che dipende dal farmaco che si sta somministrando.
La via di somministrazione intratecale, viene utilizzata in via preventiva, in pazienti che rischiano che il linfoma 
possa espandersi a livello del sistema nervoso centrale oppure in caso di linfoma già presente a livello del sistema 
nervoso centrale (cervello e midollo spinale)\cite{LYMPHACTION}.\\
Il farmaco viene somministrato all’interno del liquido cerebro-spinale in modo da oltrepassare la barriera 
emato-encefalica, la quale ha la caratteristica di consentire il passaggio di alcune sostanze, mentre blocca il 
passaggio di altre, come ad esempio i farmaci e risultano necessarie dosi piuttosto elevate, affinché un quantitativo 
sufficiente riesca ad oltrepassarla. La modalità di somministrazione avviene più comunemente mediante 
somministrazione per via endovenosa, raramente nella pratica clinica, attraverso iniezione intratecale\cite{CNS}.\\
Il farmaco più utilizzato per la profilassi endovenosa è il metotrexato, che deve essere somministrato ad alte dosi 
affinché possa superare la barriera emato-encefalica, ma può avere degli effetti negativi sulla funzionalità renale, 
di conseguenza non è indicato per pazienti con insufficienza renale, in tal caso si opta per la profilassi intratecale.
Generalmente, sono previste due o tre dosi di metotrexato a dosi elevate, ciascuna delle quali ha una durata che va 
dalle due alle quattro ore; questo trattamento può essere somministrato due o tre settimane dopo la fine del ciclo 
chemioterapico oppure come trattamento iniziale alla chemioterapia. Tra un ciclo e l’altro vengono eseguiti esami 
ematici, per monitorare i principali effetti collaterali correlati al metotrexato: neutropenia, trombocitopenia, 
riduzione della funzionalità epatica e renale. Altri effetti collaterali comprendono mucositi orali, nausea e vomito, 
che in genere passano con la fine del trattamento\cite{CNS}.\\
L’iniezione di chemioterapia direttamente nel liquido cerebro-spinale (chemioterapia intratecale) avviene mediante 
puntura lombare, ha una durata di circa venti minuti e solitamente è eseguita in sede di chemioterapia. 
La via di somministrazione intratecale evita la barriera emato-encefalica, ma solo alcuni farmaci possono essere 
somministrati in sicurezza e il metotrexato è uno di questi; in questo modo vengono somministrate dosi più elevate di 
farmaco, il quale non raggiunge tutto il corpo, ma solo alcune parti del sistema nervoso centrale e di 
conseguenza, non provoca effetti negativi sui reni. Tuttavia, questa via di somministrazione, non è del tutto 
raccomandata, per via degli effetti collaterali e delle scarse evidenze scientifiche a riguardo. 
Effetti collaterali possono comprendere: mal di testa, vertigini, febbre, intorpidimento a livello di schiena, 
collo o spalle, tremori, nausea e vomito\cite{CNS}.\\

\begin{figure}[H]
    \begin{center}
    \includegraphics[width=0.7\columnwidth]{img/Lumbar-Puncture.jpeg}
    \end{center}
    \caption[Modalità di esecuzione puntura lombare]{Modalità di esecuzione puntura lombare
    \cite{img39}}

\end{figure}

\subsection{Chemioterapia nel linfoma non-Hodgkin a basso grado di malignità o indolente}

Generalmente, per la chemioterapia ci si reca in ospedale nei giorni di trattamento, possono essere previsti degli 
esami ematici di controllo, ma non occorre l’ospedalizzazione, se non in caso di chemioterapia intensiva o in caso di 
complicanze. Lo schema terapeutico può prevedere l’assunzione, in autonomia da parte del paziente, di altri farmaci, 
che possono essere chemioterapici, oppure farmaci che servono ad alleviare eventuali effetti collaterali 
come nausea, vomito, febbre e dolore\cite{LOWGRADE}.\\
Gli schemi terapeutici vengono indicati con degli acronimi, che corrispondono alle iniziali dei farmaci compresi nello 
schema stesso. Gli schemi di terapia più utilizzati per il linfoma non-Hodgkin a basso grado di malignità sono: 
la bendamustina, somministrata in cicli di 28 giorni; l’associazione  ciclofosfamide, vincristina (nota anche come 
Oncovin) e prednisolone (CVP) oppure quella di ciclofosfamide, doxorubicina, vincristina (o Oncovin) e prednisolone 
(CHOP). Il ciclo di CVP e CHOP è di 21 giorni. A questi schemi terapeutici spesso vengono associati anticorpi 
monoclonali come rituximab o obinutuzumab. Il prednisolone in genere viene assunto per via orale, per i primi cinque 
giorni di ogni ciclo di chemioterapia.\\
Alcuni tipi di LNH a basso grado di malignità vengono trattati con altri schemi terapeutici, come DRC 
(desametasone, rituximab e ciclofosfamide); clorambucile, da solo o associato ad un anticorpo monoclonale\cite{LOWGRADE}.\\
In caso di linfoma a cellule del mantello, se c’è l’indicazione terapeutica, si può optare per la somministrazione di 
citarabina, in cicli di chemioterapia standard, o all’interno di schemi terapeutici più intensivi, come la  DHAP 
(desametasone, citarabina ad alte dosi chiamata anche Ara-C, cisplatino) oppure il “Nordic Protocol” (rituximab 
alternato ad alte dosi di CHOP e citarabina ad alte dosi) poi seguiti in genere dal trapianto di cellule 
staminali\cite{LOWGRADE}.\\
Un altro protocollo terapeutico per il linfoma a cellule del mantello non trattato e non candidabile al trapianto di 
cellule staminali, prevede l’associazione di bortezomib a R-CHP (rituximab, ciclofosfamide, doxorubicina, prednisone); 
il bortezomib può essere somministrato per via endovenosa o sottocutanea, due volte a settimana per due settimane, 
seguite da dieci giorni di riposo e si possono ricevere dai sei agli otto cicli\cite{LOWGRADE}.\\ 

\subsection{Chemioterapia nel linfoma non-Hodgkin ad alto grado di malignità o aggressivo}

Il trattamento delle forme aggressive di LNH inizia al momento della diagnosi. 
Lo schema di chemioterapia più comune per le forme aggressive di LNH è CHOP (ciclofosfamide, doxorubicina, vincristina 
o oncovin, prednisone) o R-CHOP quando è associato anche a rituximab. Ogni ciclo di CHOP ha la durata di 21 giorni e 
possono essere somministrati da tre a sei cicli di tale trattamento, in base al tipo di linfoma, alla sua estensione 
e a quanto la malattia risponde alla terapia.
Schemi di trattamento più intensivi sono DA-EPOCH (dose-adjusted etoposide, prednisolone, vincristina, ciclofosfamide 
e doxorubicina); CODOX-M (ciclofosfamide, vincristina, doxorubicina e metotrexato), talvolta alternato con lo schema 
IVAC (ifosfamide, etoposide o VP-16 e citarabina o Ara-C); MATRix (metotrexato, citarabina o Ara-C, 
thiotepa e rituximab)\cite{HIGHGRADE}.\\

\section{Complicanze correlate alla chemioterapia}

Le complicanze e gli effetti collaterali della chemioterapia dipendono dall’intensità e dal tipo di trattamento, da 
quali farmaci vengono utilizzati, dalla parte del corpo interessata dal trattamento, dall’età del paziente, dalla 
presenza di patologie concomitanti come ad esempio diabete mellito o insufficienza renale cronica. Negli ultimi anni 
sono stati sviluppati diversi farmaci per cercare di controllare alcuni sintomi, come nausea e vomito, molti degli 
effetti collaterali sono momentanei e passano con la fine del trattamento. I benefici, in termini di remissione 
della malattia e cura, che possono arrivare dal trattamento del LNH, superano le complicanze e i fattori di rischio\cite{LLS}.\\
Gli effetti collaterali della chemioterapia, possono riguardare diversi organi e apparati.

\subsection{Complicanze a carico del sistema gastrointestinale}

Complicanze a carico del sistema gastrointestinale possono essere nausea, vomito, stipsi, diarrea, stomatiti.\\
Nausea e vomito sono causate principalmente dall’irritazione e dall’alterazione della motilità gastroenterica 
provocata dalla chemioterapia. Gli approcci per cercare di ridurre gli episodi di nausea e vomito prevedono di 
adeguare la dieta in base alle preferenze del paziente, effettuare pasti piccoli e frequenti, con cibi freddi se 
migliore per il paziente, evitare la presenza nell’ambiente di odori sgradevoli, cercare delle distrazioni, con 
immagini e musicoterapia per diminuire l’ansia, incoraggiare un’igiene orale frequente, somministrare antiemetici, 
sedativi e corticosteroidi prima e dopo il trattamento se necessario, assicurare un’adeguata idratazione prima, 
durante e dopo la chemio.\\
La stipsi è una complicanza dovuta ai farmaci, ad uno stato di disidratazione, ansia e depressione, riduzione 
dell’attività fisica. L’infermiere che accerta una condizione di stipsi, interviene informando il paziente sulla 
necessità di aumentare l’apporto di fibre, assumere 2 litri di liquidi nelle 24 ore e incrementare l’attività fisica; 
in caso di parziale o totale dipendenza optare per il presidio che renda l’evacuazione il più possibile fisiologica, 
che garantisca l’autonomia e la privacy del paziente; valutare la necessità di stimolare l’evacuazione mediante la 
somministrazione di lassativi orali.\\ 
In caso di diarrea, garantire una buona idratazione, consigliare un regime dietetico idoneo, in collaborazione con il 
dietologo, andando ad evitare cibi che irritano l’intestino (come fibre, latte, cibi troppo caldi o troppo freddi), 
monitorare segni di squilibrio elettrolitico o disidratazione e riferire al medico, valutare la necessità di 
somministrazione di antidiarroici, antispastici, ansiolitici.\\
Le stomatiti sono una complicanza infiammatoria grave, a carico della mucosa del cavo oro-faringeo. 
Segni e sintomi sono imbiancamento, eritemi, dolore, ulcerazioni della mucosa, alterazioni del gusto, xerostomia o 
scialorrea, impossibilità ad alimentarsi per difficoltà alla deglutizione e all’assunzione di cibi freddi, caldi, 
acidi e piccanti, difficoltà ad articolare la parola. È una complicanza che può aumentare il rischio infettivo, 
nonché problemi nutrizionali fino ad anoressia e cachessia, aumento delle giornate di degenza, ridotta capacità 
comunicativa, aumento del disagio del paziente, che va incontro a depressione, ansia e sconforto. 
Il ruolo dell’infermiere nella prevenzione di tale complicanza consiste nell’esaminare quotidianamente il cavo orale, 
accertando che non vi siano arrossamenti e lesioni aperte, informare il paziente sulla necessità di riferire 
eventuali dolore, bruciore o diminuzione della tolleranza verso cibi caldi o freddi. Nella prevenzione di tale 
complicanza, evitare colluttori, usare spazzolini con setole morbide e dentifrici non abrasivi; rimuovere le 
protesi e limitare il loro uso al solo momento del pasto; limitare l'assunzione di cibi molto caldi, 
piccanti e speziati.\\ % MANCA CITAZIONE

\subsection{Complicanze a carico del sistema ematopoietico}

Gli effetti collaterali della chemioterapia coinvolgono anche il sistema ematopoietico.\\
Per quanto riguarda i leucociti, si avrà una leucopenia, cioè una diminuzione del numero di globuli bianchi 
circolanti, inferiori a 4000 cell/mm3 (valori di riferimento sono tra 4000-10000 cell/mm3).\\ 
Tra i leucociti, la diminuzione del numero dei neutrofili (valori di riferimento sono tra 2000-8000 mcL) 
prende il nome di neutropenia, che può essere lieve (1000-1500 mcL), moderata (500-1000 mcL) o severa (<500 mcL). 
La neutropenia espone il paziente a rischio infettivo, è una complicanza che insorge intorno al 7°-10° giorno di 
trattamento; se intervengono infezioni gravi si somministra terapia antibiotica, antimicotica e si valuta per i 
fattori di crescita.\\ %MANCA CITAZIONE 
La neutropenia febbrile (con TC>38°C e neutrofili<500 mcL) è associata ad un rischio diretto di mortalità del 9,5\%. 
In Italia, il fattore di crescita mieloide disponibile per l’uso clinico è il G-CSF (granulocyte colony 
stimulating factor), di cui esistono quattro formulazioni: filgrastim, lenograstim, pegfilgrastim, lipegfilgrastim\cite{AIOMTOSS}.\\
L’infermiere deve attuare interventi che vadano a ridurre i rischi infettivi e le complicanze correlate alla 
neutropenia: deve adottare le corrette precauzioni per la prevenzione delle infezioni correlate all’assistenza (ICA), 
gestire correttamente gli accessi venosi, centrali e periferici ed eventuali drenaggi, monitorare la temperatura 
corporea, la comparsa di mucosite del cavo orale, rilevare precocemente segni e sintomi di sepsi, informare il 
paziente e i familiari sulle complicanze infettive e far osservare le norme comportamentali e di igiene. 
Se i neutrofili sono <1000 si attua l’isolamento protettivo del paziente, prestare attenzione ad una corretta 
igiene delle mani, usare i guanti monouso secondo indicazione, la mascherina e il camice monouso.\\ %citazione
L’anemia è la riduzione del numero di globuli rossi, segni e sintomi sono fatigue, dispnea, 
pallore, aumento della frequenza cardiaca, cute fredda\cite{LLSLOWCELLS}.\\
L’anemia provoca una riduzione della concentrazione di emoglobina, bisogna infatti  andare ad agire sulla 
causa dell’anemia e aumentare i livelli di emoglobina, talvolta si può rendere necessaria una trasfusione di sangue\cite{AMERICANLOWCELLS}.\\
Secondo quanto riportato dalla letteratura, ad oggi è raccomandato l’uso di ESA (erythropoiesis stimulating agents) 
per l’anemia indotta da chemioterapici e gli ESA disponibili in Italia sono eritropoietina alfa, eritropoietina beta, 
eritropoietina teta, darbepoietina alfa. Secondo le linee guida, i benefici che si ottengono dalla somministrazione 
degli ESA, superano i rischi, ma vanno sempre utilizzati con cautela. L’AIFA, ha approvato l’uso di tali farmaci 
in caso di anemia indotta da chemioterapia con concentrazioni di Hb < 10g/dl\cite{AIOMTOSS}.\\
Oltre alla somministrazione di ESA, per trattare l’anemia, bisogna anche correggere eventuali carenze di vitamina B12, 
acido folico e la sideropenia. La trasfusione di sangue viene valutata non solo in caso di carenza di emoglobina, 
ma considerando anche le caratteristiche del paziente (ad esempio la presenza di patologie cardiovascolari o 
polmonari pre-esistenti), e quindi la capacità di mettere in atto dei meccanismi di compensazione, la rapidità di 
insorgenza dell’anemia, la gravità e la durata. Grazie alla trasfusione di emazie, la concentrazione di emoglobina 
aumenta rapidamente, quindi è l’unica opzione in caso di pazienti che devono correggere rapidamente i livelli di 
emoglobina; l’uso delle trasfusioni è comunque restrittivo, per via della scarsa disponibilità di sangue, 
soprattutto in Italia, che pertanto deve essere usato con parsimonia\cite{AIOMTOSS}.\\
La trombocitopenia è la riduzione del numero delle piastrine, che espone il paziente a rischio emorragico. 
Segni di trombocitopenia sono: sanguinamenti eccessivi dal naso o dalle gengive, petecchie soprattutto su 
gambe e caviglie, ematomi, presenza di ematuria, mal di testa, vertigini e debolezza\cite{LLSLOWCELLS}.\\
Bisogna pertanto evitare sanguinamenti anche minimi, tagli e ferite; prestare attenzione durante l’igiene orale a non 
provocare sanguinamenti delle gengive, in tal caso, sciacquare con acqua ghiacciata, utilizzare lozioni e balsami 
per la pelle secca e le labbra screpolate; quando si tossisce o ci si soffia il naso, non farlo con forza; 
usare solo rasoi elettrici per evitare tagli accidentali; non assumere farmaci come aspirina o ibuprofene, 
che aumentano il rischio di sanguinamento, se non prima di aver consultato il medico; evitare sport di contatto\cite{AMERICANPLATELET}.\\

\subsection{Alopecia}

La caduta di capelli e peli è causata dall’interferenza del trattamento chemioterapico (ma anche radioterapico) con 
il ciclo di riproduzione cellulare dei follicoli piliferi. Non tutti i farmaci determinano alopecia e non tutti con 
la stessa intensità: ciò dipende dal tipo di farmaco, dal dosaggio e da caratteristiche proprie della persona. 
L’esordio può essere dopo alcuni giorni o settimane, in genere non è doloroso, la reversibilità si ha dopo 3-6 mesi 
dalla fine della chemioterapia; i capelli possono ricrescere ricci, mentre il cambio di colore è raro. 
Sebbene sia un sintomo passeggero, l’alopecia è tra i più temuti per il cambiamento dell’immagine corporea, 
per la percezione di sé e per il richiamo costante alla patologia oncoematologica nella vita sociale. 
Per quanto riguarda i trattamenti, non ci sono dei trattamenti efficaci, vi sono però due approcci, che usano il 
principio della vasocostrizione periferica, per tentare di frenare questo sintomo:
l’uso di termo-cuffie ghiacciate o l’immissione di aria fredda sul cuoio capelluto.\\
Gli interventi infermieristici, prevedono l’educazione del paziente, consigliare il taglio preventivo dei capelli, 
comprare una parrucca dello stesso colore dei capelli o un copricapo, la caduta dei capelli è ritardata se si evita 
di spazzolarli con forza ed evitando l’uso di asciugacapelli. Consigliare comunque l’attivazione di un supporto 
psicologico, per ridurre il trauma correlato alla caduta dei capelli, fornire strategie di supporto e favorire 
il processo di accettazione. %citazione

\subsection{Infertilità}

I farmaci chemioterapici, possono andare a incidere sulla fertilità della donna e dell’uomo. Per la donna infertilità 
significa non riuscire a concepire il bambino naturalmente e a portare a termine la gravidanza, 
per l’uomo, vuol dire non riuscire a concepire il bambino attraverso attività sessuale naturale\cite{AMERICANFERTILITY}.\\
Nella donna, il trattamento di chemioterapia, può incidere sulla regolarità del ciclo mestruale, fino anche ad una 
menopausa precoce in alcuni casi. Alcuni farmaci utilizzati per il trattamento del LNH possono causare infertilità, 
per alcune persone, anche basse dosi di farmaco possono provocare questa complicanza; ci sono alcuni chemioterapici 
per cui non si può prevedere se renderanno il paziente infertile, perché dipende dal tipo di farmaco, 
dalla sua dose totale e dell’età, nelle donne\cite{UKFERTILITY}.\\
Altri farmaci inoltre possono determinare un’infertilità temporanea, quindi dopo un tempo che va dai sei mesi fino ad un anno, dalla fine del trattamento, la fertilità 
può ripristinarsi. Se invece si è sottoposti ad un trapianto di cellule staminali, quasi certamente si sarà sterili\cite{UKFERTILITY}.
Per preservare gli spermatozoi o le cellule uovo, prima di intraprendere il trattamento, essi vengono raccolti e 
congelati per essere utilizzati in futuro. È un processo più complicato per la donna, piuttosto che per l’uomo, 
in quanto la donna deve assumere ormoni per circa due settimane, che vadano a stimolare la produzione di un numero 
più elevato di ovociti rispetto al solito. Le cellule uovo possono poi essere conservate fino al momento dell’utilizzo.\\

Altri sintomi correlati alla chemioterapia sono lo squilibrio elettrolitico, neuropatie periferiche, 
iperpigmentazione cutanea, ipoacusia, cirrosi epatica.\\

\section{Manipolazione dei farmaci chemioterapici}

Secondo quanto riportato dall’ISS (Istituto Superiore di Sanità), gli esposti ad antiblastici in ambiente 
ospedaliero, risultano a rischio, nel breve e nel lungo termine, di sviluppare tumori (tra cui leucemie e 
linfomi non Hodgkin) e danni a carico dell’apparato riproduttivo, causati dall’esposizione modesta e protratta nel 
tempo; è stato comunque confermato da diversi studi come attraverso l’attuazione di misure di sicurezza, 
si ha l’annullamento o comunque la notevole riduzione di tali effetti\cite{ISSESPO}.\\
Monitorare con attenzione, l’esposizione del personale sanitario, che è coinvolto nella manipolazione e 
somministrazione dei farmaci chemioterapici è necessario, visto l’ampio uso di tali farmaci 
anche in ambito domestico e la continua immissione nel mercato di nuovi principi attivi.\\
Uno degli obiettivi della sorveglianza sanitaria è di identificare precocemente, eventuali alterazioni dello stato 
di salute degli operatori sanitari, esposti a manipolazione dei farmaci antiblastici. Per fare ciò, 
bisogna garantire ai lavoratori degli accertamenti periodici e valutare il rischio di esposizione professionale\cite{ISSESPO}.\\
I dispositivi di protezione individuale (DPI) da utilizzare, comprendono guanti di lattice pesanti, lunghi e indossati 
sopra il polsino del camice, da sostituire dopo 30 minuti, ad ogni cambio paziente e tassativamente in caso di 
lacerazione; prima di indossare i guanti e dopo averli rimossi, eseguire un accurato lavaggio delle mani. 
Il camice deve essere sterile, monouso, a maniche lunghe e con polsini, in TNT o analoghi; la mascherina facciale 
filtrante di tipo FFP3; la cuffia monouso in TNT per contenere i capelli, le sovrascarpe monouso e impermeabili; 
gli occhiali o la visiera con protezioni laterali. In caso di contaminazione accidentale dei DPI monouso,
l’operatore deve rimuoverli subito e procedere al loro smaltimento negli appositi contenitori. %citazione
I contenitori per i rifiuti comprendono i contenitori a tenuta per il trasporto dei CTA, 
contenitori rigidi per taglienti. %citazione
Le sorgenti di esposizione sono principalmente la via inalatoria e percutanea; la contaminazione per via oculare 
causata da spruzzi o la via digestiva causata da ingestione, sono occasionali e incidentali\cite{FNOPI}.\\ 
L’esposizione professionale dei lavoratori coinvolge diverse fasi della manipolazione. Durante la fase di 
immagazzinamento, confezioni di farmaci non integre rappresentano un rischio per l’operatore addetto al ricevimento e 
allo stoccaggio nella farmacia e nei reparti oncologici. Se il farmaco deve essere manipolato prima di somministrarlo 
al paziente, si può provocare inquinamento atmosferico quando si apre la fiala, si estrae l’ago dal flacone, 
si inserisce il farmaco nella siringa o fleboclisi, si toglie l’aria dalla siringa per dosare il farmaco\cite{FNOPI}.\\ 
Durante la somministrazione si è esposti a rischio cutaneo in caso di stravasi di liquido dai deflussori, 
dai flaconi o dalle connessioni, contatto con materiali biologici o biancheria dei pazienti. 
L’esposizione del personale si ha anche durante le fasi di smaltimento e nella manutenzione delle cappe\cite{FNOPI}.\\
I principali effetti dei farmaci neoplastici sui lavoratori possono essere acuti (reazione irritante, vescicante e 
allergica per contatto con cute e mucose) e cronici (effetti sistemici sporadici come vertigini, cefalea, vomito e 
alopecia). È fondamentale formare adeguatamente il personale affinché comprenda la natura del rischio cui è esposto, 
come evitarlo o ridurlo al minimo. Bisogna garantire qualità è sicurezza, riducendo al minimo il rischio di errore 
nella terapia farmacologica, in quanto i farmaci chemioterapici hanno un basso indice terapeutico. I dosaggi devono 
essere personalizzati per ridurre la tossicità, la somministrazione avviene in sequenze definite, 
in associazione a farmaci che vadano a ridurre gli effetti collaterali. %citazione

\section{Gli accessi venosi centrali}

Nel paziente oncoematologico, è di fondamentale importanza la presenza di un accesso venoso. I dispositivi per 
l’accesso venoso possono essere periferici e centrali e la scelta del dispositivo più adatto dipende da una serie di 
fattori: il paziente oncologico è sottoposto a regimi di polichemioterapia, i cui farmaci risultano aggressivi e 
lesivi per vene di piccolo calibro, sono pazienti che necessitano di prelievi ematici frequenti, che infondono fluidi 
ed elettroliti, ma possono rendersi necessarie anche trasfusioni di sangue ed emocomponenti, terapie antalgiche, 
palliative e nutrizione parenterale (NPT).\\
L’infermiere deve essere adeguatamente formato e preparato in quanto svolge un ruolo fondamentale nella gestione 
degli accessi venosi, nell’informazione ed educazione del paziente.\\ %citazione
Si parla di accesso venoso periferico o catetere venoso periferico (CVP) quando la punta del catetere, 
indipendentemente dal sito di accesso, non arriva in prossimità della giunzione tra vena cava superiore ed atrio destro. 
I dispositivi periferici sono le cannule metalliche (ad esempio i “Butterfly”) non adatte ad infusioni continue, 
presentano un rischio elevato di trombosi, stravaso e infezione; cannule corte (3-6 cm), cannule lunghe o 
mini-midline (6-15 cm) e cateteri midline (>15 cm)\cite{GAVECELTCVC}.\\
Secondo le linee guida, sono consentite mediante catetere venoso periferico, infusioni di soluzioni con pH 5-9, 
farmaci con osmolarità < 500-600 mOsm/L, soluzioni nutrizionali con osmolarità < 800-900 mOsm/L, farmaci non 
vescicanti e non urticanti. Inoltre, secondo le indicazioni più recenti, non è più raccomandata la sostituzione 
degli agocannula dopo 96 ore, se il dispositivo funziona e non presenta segni di complicanze, 
può essere lasciato in sede anche più a lungo\cite{GAVECELTCVC}.\\
Per rilevare in modo precoce le complicanze, il personale infermieristico deve ad ogni turno ispezionare 
l’exit-site del catetere e verificare il corretto funzionamento dello stesso\cite{GAVECELTracc2021}.\\

L’accesso venoso centrale o catetere venoso centrale (CVC) è così definito quando la punta del catetere arriva in 
vena cava superiore, in atrio destro o in vena cava inferiore. Dispositivi per l’accesso centrale sono PICC 
(peripherally inserted central catheter), CICC (central inserted central catheter), 
FICC (femorally inserted central catheter), dispositivi totalmente impiantabili come port-a-cath o port, PICC-port\cite{GAVECELTracc2021}.\\
I farmaci somministrati per via centrale possono avere un pH>9 o pH<5, le soluzioni possono essere ipertoniche, con 
osmolarità>800 mOsm/L, farmaci vescicanti o irritanti, nutrizione parenterale, %citazione
questo grazie all’elevato flusso ematico che va a diluire tali farmaci e quindi riduce 
la potenziale lesività sull’endotelio\cite{GAVECELTracc2021}.\\
Grazie all’accesso centrale possono essere eseguiti prelievi ematici frequenti ed emodialisi; bisogna precisare 
invece che per quanto riguarda il monitoraggio emodinamico, esso è consentito da cateteri 
la cui punta arriva in atrio destro, ma non da quelli che arrivano in vena cava inferiore\cite{GAVECELTracc2021}.\\

\begin{figure}[H]
    \begin{center}
    \includegraphics[width=0.6\columnwidth]{img/picc.jpeg}
    \end{center}
    \caption[PICC e midline a confronto]{PICC e midline a confronto
    \cite{img40}}

\end{figure}

\begin{figure}[H]
    \begin{center}
    \includegraphics[width=0.5\columnwidth]{img/CVC.png}
    \end{center}
    \caption[Il CVC (catetere venoso centrale)]{Il CVC (catetere venoso centrale)
    \cite{img41}}

\end{figure}

\begin{figure}[H]
    \begin{center}
    \includegraphics[width=0.6\columnwidth]{img/PORT.jpeg}
    \end{center}
    \caption[Il PORT]{Il PORT
    \cite{img42}}

\end{figure}

La scelta del presidio più adatto è di fondamentale importanza. L’infermiere svolge un ruolo centrale 
nell’informazione ed educazione del paziente e del caregiver nella scelta del presidio, che deve mirare a preservare 
il patrimonio venoso del paziente, garantendo la somministrazione del trattamento in modo responsabile. 
Nella scelta del presidio bisogna considerare la sicurezza del paziente, l’efficacia del dispositivo, il rapporto 
costi-efficacia, riduzione dei tempi di cura, migliore compliance del paziente. Inoltre, bisogna valutare una serie 
di fattori correlati al paziente:  l’età , il tipo di  neoplasia, il performance status, la comorbilità, la presenza 
di eventuali allergie. Bisogna sempre, per quanto possibile, cercare di agire soddisfando le preferenze del paziente, 
ed accertarsi sulla presenza o assenza di un caregiver. Per quanto riguarda i fattori correlati al patrimonio 
venoso periferico, bisogna effettuare una valutazione dello stesso, analizzando le vene di entrambi gli arti superiori,
documentando le caratteristiche della vena per ogni arto (calibro, palpabile, turgida, decorso lineare) ed eventuali 
alterazioni. Infine, si considerano i fattori correlati alla terapia: bisogna consultare la scheda tecnica del farmaco,
per avere informazioni su pH, osmolarità, valutazione del potenziale danno tissutale in caso di stravaso, valutare la 
scheda di terapia per quanto riguarda l’intento terapeutico, durata della terapia e frequenza delle somministrazioni\cite{AIOMCVC}.\\

\subsection{Il PORT}

Il port è un dispositivo totalmente impiantato sottocute, costituito da un catetere collegato ad un reservoir\cite{AIOMCVC}.\\
È uno dei dispositivi ad uso extra-ospedaliero che può rimanere in sede per lunghi periodi (> 4 mesi)\cite{GAVECELTracc2021}.\\
L’accesso al port, avviene, previa disinfezione, mediante puntura percutanea della membrana del reservoir, con 
l’utilizzo dell’ago di Huber\cite{AIOMCVC}.\\
La particolarità dell’ago di Huber è che la punta risulta piegata, questa caratteristica fa sì che il 
setto del port non venga danneggiato e può essere utilizzato per periodi lunghi, %citazione
consentendo fino a 2000-3000 accessi da tale dispositivo\cite{AIOMCVC}.\\

\begin{figure}[H]
    \begin{center}
    \includegraphics[width=0.6\columnwidth]{img/port-a-cath-picc.jpeg}
    \end{center}
    \caption[Apertura del port mediante puntura percutanea con ago di Huber]{Apertura del port mediante puntura percutanea con ago di Huber
    \cite{img43}}

\end{figure}

\begin{figure}[H]
    \begin{center}
    \includegraphics[width=0.5\columnwidth]{img/port2.jpeg}
    \end{center}
    \caption[Puntura percutanea del port, come impugnare l’ago di huber e mantenere ben saldo il port.]{Puntura percutanea del port, come impugnare l’ago di huber e mantenere ben saldo il port.
    \cite{img44}}

\end{figure}

\begin{figure}[H]
    \begin{center}
    \includegraphics[width=0.5\columnwidth]{img/agohuber.jpeg}
    \end{center}
    \caption[Ago di Huber]{Ago di Huber
    \cite{img45}}

\end{figure}

Prima di inserire l’ago di Huber, bisogna effettuare la disinfezione 
della cute con clorexidina al 2\% in alcol isopropilico al 70\%. L’ago di Huber va inserito con i guanti sterili\cite{GAVECELTracc2021}.\\
Dopo l’inserzione dell’ago di huber, si applica una medicazione, la quale dovrà essere sterile, trasparente, 
in modo da rendere visibile il punto di inserzione e semipermeabile, deve fissare e proteggere il punto di inserzione 
da eventuali contaminazioni esterne. Prima di iniziare l’infusione bisogna verificare la pervietà del port, 
dopo l’infusione invece bisogna effettuare l’irrigazione (flush) con soluzione fisiologica con tecnica pulsata, 
utilizzando siringhe monouso di tipo luer-lock da 10 ml preriempite; nell’eseguire il flush, per evitare il reflusso 
di sangue, si cerca di mantenere una pressione positiva lasciando un quantitativo di 0,5-1 ml di fisiologica nella 
siringa. Per quanto riguarda il lock, secondo le linee guida AIOM, non vi sono evidenze sufficienti per raccomandare 
soluzioni eparinate o con fisiologica, in quanto gli studi effettuati hanno ottenuto risultati sovrapponibili\cite{AIOMCVC}.\\
Quando non utilizzato, bisogna effettuare un periodico flush and lock del port (ogni quattro settimane), l’ago di Huber
invece deve essere rimosso quando non più necessario e comunque sostituito dopo 7 giorni.I sistemi totalmente 
impiantabili sono soggetti a complicanze ostruttive di difficile risoluzione, motivo per cui si dovrebbe evitare di 
eseguire prelievi ematici di routine, emotrasfusioni, infusioni di nutrizioni parenterali con emulsioni lipidiche e 
di mezzo di contrasto\cite{GAVECELTracc2021}.\\
Il port è un dispositivo che ha come vantaggio di preservare l’immagine corporea del paziente, soprattutto in pazienti 
giovani, che possono svolgere attività fisica, possono nuotare e fare il bagno, è più comodo da gestire ed ha un 
rischio minore di infezione.\\ 
Gli svantaggi sono che è necessario un ago apposito per non danneggiare la membrana, il paziente può avvertire dolore 
durante l’inserimento e la rimozione dell’ago di huber, si possono provocare danni cutanei cronici e sanguinamenti 
dopo la puntura, l’ago può dislocarsi dal reservoir e ciò può provocare stravaso, l’operatore deve essere 
adeguatamente formato ed è esposto a maggior rischio di puntura accidentale.\\ %citazione

\subsection{Il PICC}

Il PICC in oncologia viene utilizzato con le funzioni proprie degli accessi venosi centrali. 
Tra i vantaggi del PICC abbiamo un minor rischio di sepsi sistemiche, meno rischi rispetto alla venipuntura 
centrale e costi ridotti, la rimozione è semplice. Tra gli svantaggi vi sono le possibili tromboflebiti locali, 
malposizionamento, è richiesta esperienza e conoscenza da parte dell’operatore, è richiesto l’rx di controllo per 
verificare il corretto posizionamento, ha una durata limitata nel tempo rispetto ad altri presidi. %CITAZIONE
Per la corretta gestione del PICC, è importante che la medicazione venga sostituita ogni 5-7 giorni o prima se risulta
sporca o staccata, documentando data e ora, le condizioni del catetere e della cute. Sostituzioni più frequenti non 
riducono il rischio di infezione, al contrario, aumentano il rischio di contaminazione dell’exit-site. La medicazione 
deve essere sterile, trasparente, semipermeabile, protettiva, facile da applicare e rimuovere e deve garantire 
il comfort del paziente\cite{AIOMCVC}.\\
Nel caso in cui il sito di inserzione presenti eccessive secrezioni o sanguinamento, utilizzare una medicazione con 
garza sterile e cerotto, da cambiare al massimo dopo 48 ore o comunque quando risulta bagnata o sporca, sostituendola 
appena possibile con una medicazione semipermeabile trasparente in poliuretano, con TNT perimetrale e sistema 
sutureless integrato, con clorexidina al 2\% a lento rilascio.\\ 
L’exit-site del catetere è la fonte primaria di ingresso di microrganismi, pertanto bisogna sempre procedere 
rispettando le regole di asepsi; esso deve essere ispezionato quotidianamente, per evidenziare eventuali arrossamenti, 
secrezioni, edemi, motivo per cui la medicazione deve essere trasparente\cite{AIOMCVC}.\\
Per la disinfezione, si utilizza la clorexidina al 2\% in soluzione alcolica, in caso di indicazioni di allergia del 
paziente, si può utilizzare iodopovidone al 7\% o al 10\%. È di fondamentale importanza rispettare i tempi di
contatto e asciugatura del disinfettante, che è di 30 secondi per la clorexidina e 1,5-2 minuti per lo iodopovidone.
Il picc deve essere fissato alla cute utilizzando dei sutureless device, (le suture aumentano il rischio di infezioni) 
che devono sempre rendere visibile l’exit-site e devono essere sostituiti contestualmente al cambio della medicazione 
o comunque ogni 7 giorni, prestando attenzione a non dislocare accidentalmente il catetere. 
Per fissare la porzione esterna del catetere, non utilizzare dei bendaggi, perché non renderebbero visibili eventuali 
segni di infezione e potrebbero anche risultare compressivi\cite{AIOMCVC}.\\
Al termine dell’infusione della terapia, bisogna effettuare il lavaggio (flush) e la chiusura (lock) del PICC.
Il lavaggio viene effettuato utilizzando una siringa preriempita di tipo luer-lock con soluzione fisiologica, 
iniettata con tecnica pulsata e chiudendo il sistema con una pressione positiva. La chiusura del PICC viene effettuata 
generalmente con 10 mL di soluzione fisiologica; nel caso in cui si utilizzi una soluzione eparinata, la 
concentrazione di eparina deve essere di 10 UI/mL, ma non vi sono sufficienti evidenze che raccomandano l’uso 
dell’una o dell’altra soluzione. 
Il lock, deve essere effettuato con un volume di soluzione pari al volume interno del sistema, più il 20\%.\\
Le terminazioni del catetere devono essere protette mediante needle-free connectors con attacco luer-lock a pressione 
neutra (immagine n.39), da disinfettare prima dell’utilizzo, per evitare che microrganismi possano penetrare 
all’interno. La disinfezione può avvenire in modo attivo oppure passivo. La disinfezione attiva prevede lo 
strofinamento con disinfettante, rispettando i tempi di azione e asciugatura del disinfettante stesso; 
lo strofinamento deve essere vigoroso, della durata di 5-15 secondi. I disinfettanti utilizzabili sono 
l’alcol isopropilico al 70\%, lo iodopovidone o la clorexidina al 2\% in soluzione alcolica\cite{AIOMCVC}.\\
La disinfezione passiva, prevede l’utilizzo di dispositivi chiamati port protectors, i quali sono cappucci contenenti 
una spugna imbevuta di alcol isopropilico al 70\%. Questi dispositivi hanno mostrato una riduzione del rischio di 
infezione intraluminale. Bisogna comunque fare sempre riferimento alle indicazioni del produttore per quanto 
riguarda i tempi di disinfezione e di garanzia della protezione\cite{AIOMCVC}.\\

\begin{figure}[H]
    \begin{center}
    \includegraphics[width=0.5\columnwidth]{img/needlefree.png}
    \end{center}
    \caption[Esempi di needle-free connectors utilizzati nella pratica clinica]{Esempi di needle-free connectors utilizzati nella pratica clinica
    \cite{img46}}

\end{figure}

\begin{figure}[H]
    \begin{center}
    \includegraphics[width=0.5\columnwidth]{img/portprotectors.jpg}
    \end{center}
    \caption[Port protector]{Port protector
    \cite{img47}}

\end{figure}

\subsection{Il PICC-PORT}

Il PICC-PORT è un device facente parte dei cateteri venosi centrali, costituito da un catetere e da una camera con 
setto perforabile mediante ago di Huber. Questo accesso venoso presenta le caratteristiche del PORT in quanto la 
camera è impiantata nel sottocute, ma come il PICC è inserito in una delle vene dell’avambraccio e la punta del 
catetere arriva in prossimità della giunzione cavo-atriale\cite{CRO}.\\
Il picc-port o arm port è sempre più impiegato in ambito oncologico per le terapie infusionali antiblastiche, 
antalgiche, per l’infusione di anticorpi monoclonali e per terapie di supporto nutrizionale\cite{GAVECELTPICCPORT}.\\
Negli ultimi anni, grazie alla progressione delle tecniche chirurgiche, all’utilizzo della tecnica ecoguidata, 
l’uso di dispositivi di dimensioni sempre più piccole, fatti con materiale maggiormente compatibile, ha fatto sì che 
si riducessero le complicanze nell’immediato post-operatorio come pneumotorace, puntura dell’arteria, 
ematomi del collo. Il picc-port viene pertanto inserito con la stessa tecnica del picc (puntura venosa ecoguidata), 
ma apporta i vantaggi funzionali ed estetici di un dispositivo sottocutaneo\cite{MERLICCO}.\\
Le complicanze correlate al PICC-PORT sono ematoma, infezione sottocutanea, sepsi, deiscenza cutanea, occlusione del 
catetere, trombosi venosa. 
Per la scelta del device più adatto bisogna valutare sia il diametro del braccio che il calibro della vena 
(il calibro del catetere deve essere un terzo del diametro della vena). In caso di braccio con diametro inferiore a 
35 cm si opta per un micro-port, in caso di diametro del braccio superiore a 35 cm si sceglie un 
mini-sitimplant\cite{GAVECELTPICCPORT}.\\

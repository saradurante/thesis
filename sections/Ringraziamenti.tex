\chapter*{Ringraziamenti}
\addcontentsline{toc}{chapter}{Ringraziamenti}

A Linda, ho avuto l'onore di incontrarti durante uno dei miei periodi di tirocinio, è stato per me un privilegio poter
stare al tuo fianco in reparto e nella stesura di questa tesi. Grazie, perchè quello che mi hai trasmesso umanamente e
professionalmente è impagabile, sei stata più di una semplice Professoressa, un punto di riferimento,
una fonte inesauribile di conoscenze e per me, sarai sempre un esempio da seguire nell'agire professionale.\\
Ringrazio i miei genitori, per avermi assecondata nelle mie scelte universitarie, per avermi sostenuta e supportata,
anche quando tutto andava male, non mi avete mai fatta sentire sbagliata.\\
A mio padre, mi hai insegnato che il mestiere del genitore è il più complicato di tutti,
che nella vita si può sbagliare, ma bisogna sempre porgere l'altra guancia. Grazie perchè nonostante la nostra situazione,
non mi è mancato niente.\\
A mia madre, per esserci sempre stata per me, so quanto hai sofferto, nonostante ciò non ti sei mai abbattuta di fronte a 
niente, mi hai trasmesso l'importanza di avere il sole dentro, anche nelle giornate più grigie, ad essere forte e indipendente. 
Non importa quanto io sia lontana, sapremo sempre essere vicine.\\
A Gaetano, per tutto ciò che fai ogni giorno; ho una situazione familiare
particolare o speciale, come piace dire a me, ma tu non mi hai mai fatto sentire esclusa, in difetto o estranea,
questo non è scontato e ti ringrazio; sono sicura che persone così al mondo ce ne siano davvero poche.\\
A Cristian, ti ringrazio per darmi quella leggerezza di cui tutti abbiamo
bisogno a volte, quel lato bambino, che ci permette di ridere e giocare spensierati; ovunque la vita ci condurrà, 
ci saremo sempre l'uno per l'altra, ti voglio un bene infinito.\\
A nonna Lina, nonna Rosa e nonno Pasquale, vi ringrazio per avermi trasmesso la forza di non mollare mai e di andare
avanti, l'amore e l'affetto; mi avete insegnato che amare significa accettarsi per come si è. Vi
ringrazio semplicemente per esserci, siete il mio punto di riferimento, il mio rifugio sicuro. A voi dedico i miei
traguardi più importanti, spero di avervi resa orgogliosa.\\
A Maria, hai ispirato questa tesi, la vita con te è stata ingiusta, però quando ripenso a te non ho in mente
la sofferenza, ma la tua risata dolce e quel tuo modo di chiamarmi ogni volta che arrivavo a casa tua. Non mi sono mai 
sentita di troppo e di questo ti ringrazio.\\
A mia zia Tiziana, per me sarai sempre la mia "Titti", grazie per esserci sempre per me, per supportarmi, per avermi
insegnato che anche di fronte a mille difficoltà, si stringono i denti e si va avanti.\\
A Chiara e Alessia, le mie fan numero uno, vi ringrazio per il vostro affetto, siamo cugine, ma siete un po' le mie 
sorelline acquisite, in ogni circostanza ci saremo sempre l’una per l’altra.\\
A tutti voi, anche se non lo dico abbastanza, vi voglio bene.\\
A tutti i colleghi che hanno fatto parte del mio percorso universitario, 
Veronica, Lucrezia, Valeria, Beatrice, Mariasara, Vittorio e Rosario, avete reso le giornate di studio meno 
pesanti, siete riusciti a strappare un sorriso nei momenti più difficili.\\
Con Veronica e Lucrezia abbiamo iniziato questo percorso insieme, sono felice di terminarlo insieme. 
Siamo state colleghe e coinquiline, a contatto h24 non è stato sempre facile, ne abbiamo passate tante
tra esami, tirocinio e convivenza, ma ce l’abbiamo fatta. Non ci sono state brutte giornate che non abbiamo 
potuto risollevare con qualche battuta e un buon calice di vino.\\
A Veronica, quando ci siamo conosciute, dopo dieci minuti mi sembrava di conoscerti già da 
una vita, sei genuina, pura e autentica. Non ho mai avuto una migliore amica, in te l’ho trovata. Questi anni universitari 
non sono stati semplici, grazie a te non mi sono mai sentita sola, so di poter sempre contare su di te, ci sei sempre 
stata quando ho avuto bisogno di sfogarmi e c’eri anche quando non avevo proprio voglia di parlare. 
Ti ringrazio davvero, perchè senza di te, nulla sarebbe stato lo stesso, 
sei una persona splendida, riesci a portare l’arcobaleno anche nelle giornate più grigie. 
Ti ringrazio e ti chiedo scusa, perchè io non sono una persona semplice, soprattutto nei momenti no. Però voglio dirti e 
spero tu lo ricorderai sempre, di non sentirti mai inferiore a nessuno, sei più forte di quello che immagini. 
Ti voglio bene.\\
Irene, quando sei andata via da casa Nannizzi la tua mancanza si è sentita fin da subito, 
tu sempre con la battuta pronta in ogni momento, quando tutto era pesante, riuscivi a dare leggerezza e armonia.\\
Valeria, se qualcuno mi avesse detto che dopo il periodo di tirocinio in pronto soccorso saremmo diventate amiche, 
non gli avrei mai creduto. Sei stata la scoperta più grande, quante risate mi hai fatto fare in questi mesi, 
hai davvero un cuore tenero.\\
Mariasara, abbiamo stretto il nostro rapporto in un periodo di tirocinio intenso, che grazie a te porterò sempre 
nel cuore. Non sei stata una semplice collega, ma una confidente nei momenti duri.\\
Dolcissima Beatrice, a te devo il ricordo spensierato di un'estate Senese trascorsa tra passeggiate, cene a casa e 
uscite, che non dimenticherò mai.\\
Ci tengo a ringraziare tutti i tutor clinici con cui ho avuto modo di lavorare in questi anni nei vari periodi di 
tirocinio. Vi ringrazio per avermi trasmesso la professionalità, la passione per questa professione, ogni insegnamento 
che mi avete dato è stato fondamentale alla mia crescita personale e professionale, se sono la persona che sono 
oggi è anche grazie a voi.\\
Per ultimo, ma non per importanza, ringrazio il mio fidanzato Vincenzo. 
Grazie perché sei la persona che più di tutte crede in me, riesci a credere in me prima ancora che ci 
creda io. Quando non mi ritengo abbastanza, quando penso che qualcosa per me sia impossibile, per te è sempre possibile. 
Mi sei stato accanto, come sempre quando ne ho bisogno, anche durante la stesura di questa tesi, supportandomi e sopportandomi 
in ogni momento di difficoltà. Ti ringrazio per tutte le volte in cui sono giù di 
morale o un po' troppo su di giri e riesci con un abbraccio e le tue battute a sdrammatizzare 
e a farmi ridere. Ti chiedo scusa perchè so che in quei momenti non sono una persona facile.
Siamo due persone diverse, ma nella diversità riusciamo a combaciare, mi fai sentire amata e sei sempre il mio porto sicuro. 
Non so cosa il futuro ci riserva, ma abbiamo tanti sogni che spero realizzeremo insieme. 
Love you.\\
Infine ringrazio me stessa, per non aver mollato quando al primo anno mi fu detto che non sarei diventata infermiera, 
ringrazio anche quella famosa Prof. che me l’ha detto, perchè è sempre stato un motivo per non mollare. 
Sono una persona emotiva, ma anche molto testarda, quando mi metto in testa qualcosa, sono determinata a portarla a termine; 
mi auguro di continuare ad esserlo per raggiungere i miei obiettivi e per affrontare le difficoltà della vita.\\ 
Che questo non sia un punto di arrivo ma solo un punto di partenza.\\ 
Ad maiora, semper. 


